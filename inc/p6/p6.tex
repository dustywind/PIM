
\chapter{PIM Praxisteil 06}

\section{Foo}

\subsection{p6 Aufgabe S. 2}

Das Management von Wissen wird h"aufig auf den Aspekt der Informationstechnologie reduziert.
Ein ganzheitlicher Ansatz des Wissensmanagements umfasst jedoch weitere Gesichtspunkte.\\

\noindent
$a)$ Nennen Sie bitte drei weitere Gestaltungsaspekte des Wissensmanagements.\\

\begin{quote}
\end{quote}

\noindent
$b)$ Beschreiben Sie bitte f"ur jeden der genannten Aspekte eine beispielhafte Ma"snahme und erl"autern Sie, welche Kernaktivit"aten des Knowledge-Management-Prozesses dadurch unterst"utzt werden.

\begin{quote}
\end{quote}

\noindent
$c)$ Beschreiben Sie bitte, weshalb Knowledge-Management ein Gesch"aftsprozess ist.

\rowcolors{1}{LightGray}{White}
\begin{tabular}{ l l l } % evtl. 5 Zeilen
    \hline
    \rowcolor{LightSlateGrey}
    \textbf{Gestaltungsaspekt} & \textbf{Ma"snahme} & \textbf{Kernaktivit"at}
\end{tabular}
\begin{quote}
\end{quote}

\subsection{p6 Aufgabe S. 4}
Die Nutzung von Enterprise 2.0 Anwendungen gewinnt in Unternehmen zunehmend an Bedeutung.\\
\begin{quote}
\end{quote}

\noindent
$a)$ Welche Ziele sind allgemein mit der Einf"uhrung von Enterprise 2.0 Anwendungen verbunden?\\
\begin{quote}
\end{quote}

\noindent
$b)$ Nennen und erl"autern Sie drei Beispiele f"ur Enterprise 2.0 Anwendungen.
Welche Schritte des Kernprozesses des Wissensmanagements unterst"utzen diese Anwendungen und warum?\\
\begin{quote}
\end{quote}

\noindent
$c)$ Welche Barrieren und Probleme bestehen bei der Nutzung von Enterprise 2.0 Anwendungen im Unternehmen?

\begin{quote}
\end{quote}









