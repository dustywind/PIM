

\section{Document Type Definitions (DTD)}


\subsection{Aufgabe}
Deklarieren Sie eine Attributliste f"ur das Elemente "`Catalog\_Structure"'.
Diese Attributliste soll das Attribut "`type"' beinhalten, welches vom Typ Aufz"ahlung ist und die Werte "`root"', "`node"' oder "`leaf"' annehmen kann.
Das Attribut soll als "`required"' definiert werden.\\

\lstset{style=customXML}
\begin{lstlisting}
<!ATTLIST Catalog_Structure type (root|node|leaf) #REQUIRED>
\end{lstlisting}

\subsection{Aufgabe}
F"ugen Sie dem Element BMECat ein weiteres Attribut namens "`note"' hinzu und deklarieren Sie es als optional.\\

\lstset{style=customXML}
\begin{lstlisting}
<!ELEMENT BMECat (Header,(T_New_Catalog,note)?)>
\end{lstlisting}


\subsection{Aufgabe}

Formulieren Sie eine Document Instance, die auf die DTD "`BMECat.dtd"' referenziert und folgende Inhalte Abbildet:\\
Es handelt sich um die Beschreibung des Katalogs "`Bademoden 2005"'.
Der Katalog tr"agt die ID 200345 und liegt in deutscher Sprache in der Version 1.0 vor.

\lstset{style=customXML} % given DTD
\begin{lstlisting}
<!ELEMENT BMECat (Header, T_New_Catalog)>
<!ATTLIST BMECat version CDATA #FIXED "1.2">
<!ELEMENT Header (Catalog, Buyer, Supplier)>
<!ELEMENT Catalog (Language, Catalog_Id, Catalog_Version, (Catalog_Name)?, (Currency)?>
<!ELEMENT Language (#PCDATA)>
<!ELEMENT Catalog_Id (#PCDATA)>
<!ELEMENT Catalog_Version (#PCDATA)>
<!ELEMENT Catalog_Name (#PCDATA)>
<!ELEMENT Currency (#PCDATA)>
\end{lstlisting}

\lstset{style=customXML}
\begin{lstlisting}
<!DOCTYPE BMECat "BMECat.dtd">
<? xml version='1.0' encoding="UTF-8"?>
<BMECat version='1.2'>
    <Header>
        <Catalog>
            <Language>deutsch</Language>
            <Catalog_Id>200345</Catalog_Id>
            <Catalog_Version>1.0</Catalog_Version>
        </Catalog>
    </Header>
</BMECat>
\end{lstlisting}




\subsection{Aufgabe}
Aufgrund welcher Fehler ist das folgende XML-Dokument nicht wohlgeformt?

\lstset{style=customXML}
\begin{lstlisting}
<?xml version="1.0" encoding="ISO-8859-1"?>
<!DOCTYPE order SYSTEM "order.dtd">
<order Id="234">
    <item>
        <title>XML and Java</title>
        <price currency="EUR">100.0</price>
        <quantity>1
    </item>
    <cardinfo>
        <name>Your Name<name/>
        <expiration>04/2002</expiration>
        <number></cardinfo>5283830462320010</number>
</order>
\end{lstlisting}

Korrigiertes XML:
\lstset{style=customXML}
\begin{lstlisting}
<?xml version="1.0" encoding="ISO-8859-1"?>
<!DOCTYPE order "order.dtd">
<order Id="234">
    <item>
        <title>XML and Java</title>
        <price currency="EUR">100.0</price>
        <quantity>1</quantity>
    </item>
    <cardinfo>
        <name>Your Name</name>
        <expiration>04/2002</expiration>
        <number>5283830462320010</number>
    </cardinfo>
</order>
\end{lstlisting}


\subsection{Aufgabe}

Die Innova AG beabsichtigt, den Austausch von Rechnungsdaten mit ihren Kundne per XML abzuwickeln.
Der Entwurf einer entsprechenden Document Type Definition (DTD) f"ur die zuk"unftigen Rechnungsformulare ist bereits vorhanden:

\subsubsection{Aufgabe a)}
Die Elementtypen Ansprechpartner und Land sollen optional deklariert werden.
Auspr"agungen des Elementtyps Rechnungsposition k"onnen in einer Dokument-Instanz ein oder mehrmals vorkommen.\\

\lstset{style=customXML}
\begin{lstlisting}
<!ELEMENT Kunde (Kundennummer, Firmenname, (Ansprechpartner)?, Adresse)>
<!ELEMENT Rechnungsdaten (Rechnungsnummer, Rechnungsdatum, (Rechnungsposition)*, Nettogesamtbetrag, Mehrwertsteuer, Bruttogesamtbetrag)>
\end{lstlisting}

\subsubsection{Aufgabe b)}
Deklarieren Sie bitte ein Element vom Typ Bankverbindung mit Zeichendaten als Inhalt und ordnen Sie es an passender Stelle in das Inhaltsmodell der DTD ein.\\
\lstset{style=customXML}
\begin{lstlisting}
<!ELEMENT Bankverbindung (#PCDATA)>
<!ELEMENT Kunde (Kundennummer, Firmenname, (Ansprechpartner)?, Adresse, Bankverbindung)>
\end{lstlisting}

\subsubsection{Aufgabe c)}
Da die Bankverbindungen f"ur einen l"angeren Zeitraum konstant bleibt, schlagen Sie vor, ein Entity vom Typ BV mit dem Inhalt "`Innova AG, Kto-Nr. 1234567, BLZ 76959191, Stadtsparkasse Nuernberg"' in der DTD zu deklarieren.
Bitte formulieren Sie die Entsprechende Entitytypdeklaration.\\

\lstset{style=customXML}
\begin{lstlisting}
<!ELEMENT BV (#PCDATA)>
\end{lstlisting}

\subsubsection{Aufgabe d)}
F"ur jede Rechnung sollen jeweils Bearbeiter und Bearbeitungsstatus zugeordnet werden k"onnen.
Dies kann durch Einf"uhrung von entsprechenden Attributen realisiert werden.
Definieren Sie bitte die erforderlichen Attribute und ordnen Sie diese dem Wurzelelement des Dokuments zu.
Das Attribut Bearbeiter ist als \#REQUIRED zu deklarieren.
Das Attribut Bearbeitungsstatus kann die Werte "`ungeprueft"', "`geprueft"', oder "`freigegeben"' annehmen un dhat standardm"a"sig den Wert "`ungeprueft"'.

\lstset{style=customXML}
\begin{lstlisting}
<!ATTLIST Rechnung Bearbeiter CDATA #REQUIRED>
<!ATTLIST Rechnung Bearbeitungsstatus ("ungeprueft"|"geprueft"|"freigegeben") #FIXED "ungeprueft">
\end{lstlisting}

\subsection{Aufgabe}
Ein bekannter N"urnberger Jugendsachbuchverlag m"ochte seinen content mehrfach nutzen (Web und Print). 
Sie empfehlen dem Verlag, die Inhalte medienunabh"angig in XML-Dokumenten abzulegen.
Eine DTD beschreibt die Struktur von XML-Dokumentinstanzen.
Bitte erstellen Sie mithilfe der folgenden Angaben eine DTD zur Strukturierung eines mehrfach nutzbaren Artikels:\\

"`Jeder Artikel hat eine "Uberschrift, einen Textk"orper, einen Autor und ein Erstellungsdatum.
Der Autor hat immer einen Vor- und einen Nachnamen, optional kann ein akademischer Titel angegeben werden.
Der Textk"orper besteht aus mindestens einem Absatz.
Jeder Artikel enth"alt auch einen Copyrightvermerk.
Da dieser meistens gleich ist, soll f"ur den Inhalt ein Entity mit dem Inhalt 'Tossleff Jugendsachbuch AG' vordefiniert werden.
Da nicht jeder Artikel sowohl online als auch offline verf"ugbar sein soll, wird ein entsprechendes Attribut ben"otigt (z.B. "`type"'), das den Inhalt "`web"', "`print"' oder "`multichannel"' haben kann.
Dieses Attribut ist nicht optional."'\\

\lstset{style=customXML}
\begin{lstlisting}
<!ELEMENT Artikel (Ueberschrift, Textkoerper, Autor, Erstellungsdatum, Copyright)>
    <!ELEMENT Ueberschrift (PCDATA)>
    <!ELEMENT Textkoerper ((Absatz)+)>
        <!ELEMENT Absatz (PCDATA)>
    <!ELEMENT Autor ((Vorname)+, (Nachname)+, (Titel)?)>
        <!ELEMENT Vorname (PCDATA)>
        <!ELEMENT Nachname (PCDATA)>
        <!ELEMENT Titel (PCDATA)>
    <!ELEMENT Erstellungsdatum (PCDATA)>
    <!ELEMENT Copyright (PCDATA)>
    <!ATTLIST Artikel type ('web'|'print'|'multichannel' #REQUIRED)>
\end{lstlisting}






































































