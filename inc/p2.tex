
\section{Praxisteil 2}

\subsection{SQL}

\subsubsection{Klausuraufgabe 1a}
\textbf{\#\#\# TODO \#\#\#}

\subsubsection{Klausuraufgabe 1b}

\lstset{language=SQL}
\begin{lstlisting}[frame=L]

SELECT  BestellNr, Auslieferdatum, Name
FROM    Bestellung AS b
        JOIN Kunde AS k ON b.KundenNr = k.KundenNr
WHERE   b.Bezahlt = 0
;

\end{lstlisting}


\subsubsection{Klausuraufgabe 2a}

\lstset{language=SQL,numbers=left,numberstyle=\tiny}
\begin{lstlisting}[frame=L]
SELECT      g.Bezeichnung AS Gemuese
            , COUNT(*) AS Anzahl
FROM        Patient AS p
            JOIN ESSEN AS e ON p.ID = e.PatientenID
            JOIN Gemuese g ON g.ID = e.GemueseID
WHERE       p.Schweregrad = 'schwer'
GROUP BY    g.GemueseID
;

\end{lstlisting}

\subsubsection{Klausuraufgabe 2b}

\lstset{language=SQL}
\begin{lstlisting}[frame=L]
INSERT INTO ESSEN (PatientenID, GemueseID, Datum)
VALUES  (
    (
        SELECT  ID
        FROM    Patient
        WHERE   Vorname = 'Mathilda'
                AND Name = 'Natter'
    )
    ,
    (
        SELECT  ID
        FROM    Gemuese
        WHERE   Bezeichnung = 'Sprossen'
    )
    '2.06.2011')
;
\end{lstlisting}


\subsection{Datawarehouse}
\textbf{\#\#\# TODO \#\#\#}





