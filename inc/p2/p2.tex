
\chapter{PIM Praxisteil 02}

\section{Datenmanagement}

\subsection{p2 Aufgabe S. 23}

Stellen Sie bitte da, wie ein SQL-SELECT-Statement aufgebaut ist und welchen Operationen aus der Relationentheorie die einzelnen Bestandteile entsprechen.
\begin{itemize}
    \item SELECT $A_1$, \dots, $A_n$\\
    $\Rightarrow$ Projektion
    \item FROM $R_1$, \dots, $R_n$\\
    $\Rightarrow$ Kartesisches Produkt
    \item WHERE Pr"adikat($R_1$, \dots, $R_n$)\\
    $\Rightarrow$ Selektion
\end{itemize}

\subsection{p2 Aufgabe S. 24}

\lstset{style=customSQL}
\begin{lstlisting}
SELECT  BestellNr, Auslieferdatum, Name
FROM    Bestellung AS b
        JOIN Kunde AS k ON b.KundenNr = k.KundenNr
WHERE   b.Bezahlt = 0
;
\end{lstlisting}

Alternativ ohne "`JOIN"':
\lstset{style=customSQL}
\begin{lstlisting}
SELECT  BestellNr, Auslieferdatum, Name
FROM    Bestellung
        , Kunde
WHERE   Bestellung.KundenNr = Kunde.KundenNr
        AND Bestellug.Bezahlt = 0
;
\end{lstlisting}


\subsection{p2 Aufgabe S. 26}

\lstset{style=customSQL}
\begin{lstlisting}
SELECT      g.Bezeichnung AS Gemuese
            , COUNT(*) AS Anzahl
FROM        Patient AS p
            JOIN ESSEN AS e ON p.ID = e.PatientenID
            JOIN Gemuese g ON g.ID = e.GemueseID
WHERE       p.Schweregrad = 'schwer'
GROUP BY    g.GemueseID
;
\end{lstlisting}
Alternativ ohne "`JOIN"':
\lstset{style=customSQL}
\begin{lstlisting}
SELECT      Gemuese.Bezeichnung AS Gemuese
            , COUNT(*) AS Anzahl
FROM        Patient
            , Essen
            , Gemuese
WHERE       Patient.ID = Essen.ID
            AND Gemuese.ID = Essen.GemueseID
            AND Patient.Schweregrad = 'schwer'
GROUP BY    Gemuese.GemueseID
;
\end{lstlisting}

\subsection{p2 Aufgabe S. 27}

\lstset{style=customSQL}
\begin{lstlisting}
INSERT INTO ESSEN (PatientenID, GemueseID, Datum)
VALUES  (
    (
        SELECT  ID
        FROM    Patient
        WHERE   Vorname = 'Mathilda'
                AND Name = 'Natter'
    ) -- get the ID for the given patient
    ,
    (
        SELECT  ID
        FROM    Gemuese
        WHERE   Bezeichnung = 'Sprossen'
    ) -- get the ID for the given vegetable
    '2.06.2011')
;
\end{lstlisting}


\section{Datawarehouse}

\subsubsection{p2 Aufgabe S. 57}
Data Warehouses erm"oglichen im Allgemeinen so genannte Roll-Ups.
Dabei wird die Aggregationsstufe der Daten immer weiter erh"oht, so dass die Betrachtung von den Details ausgehend stetig globaler wird.
Zur Unterst"utzung dieser Operation wurde die Datenmanipulationssprache SQL um das "`Group By Rollup"'-Statement erweitert.\\\noindent
Gegeben sei die Tabelle \textit{Verk"aufe}:\\

\begin{tabular}{l l l l l}
    \rowcolor{LightSlateGray}
    \textbf{Produkt}    & \textbf{Gebiet}   & \textbf{Quartal}  & \textbf{Monat}    & \textbf{Umsatz}\\
    CD  & D & 1/14  & Jan   & 27\\
    CD  & D & 1/14  & Feb   & 13\\
    CD  & D & 1/14  & Mar   & 12\\
    VCR & D & 1/14  & Jan   & 19\\
    VCR & D & 1/14  & Mar   & 21\\
    CD  & F & 1/14  & Jan   & 20\\
    CD  & F & 1/14  & Mar   & 18\\
    VCR & F & 1/14  & Feb   & 23\\
\end{tabular}

\bigskip

\noindent
Bestimmen Sie bitte auf Basis dieser Tabelle das Ergebnis der folgenden Abfrage:\\

\lstset{style=customSQL}
\begin{lstlisting}
SELECT          Produkt, Gebiet, Quartal
                , SUM(Umsatz) AS Gesamtumsatz
FROM            Verkaeufe
WHERE           Quartal = '1/14'
GROUP BY ROLLUP ( Quartal, Produkt, Gebiet)
;
\end{lstlisting}

\bigskip

\rowcolors{0}{White}{White}
\begin{tabular}{ l l l l }
    \textbf{Produkt} & \textbf{Gebiet} & \textbf{Quartal} & \textbf{Gesamtumsatz}\\\hline
    \rowcolor{NavyBlue!10}  CD  & D & 1/14  & 52\\
    \rowcolor{NavyBlue!10}  VCR & D & 1/14  & 40\\
    \rowcolor{NavyBlue!10}  CD  & F & 1/14  & 38\\
    \rowcolor{NavyBlue!10}  VCR & F & 1/14  & 23\\\hline
    \rowcolor{NavyBlue!20}  CD  &   & 1/14  & 90\\
    \rowcolor{NavyBlue!20}  VCR &   & 1/14  & 63\\\hline
    \rowcolor{NavyBlue!30}      &   & 1/14  & 153\\
\end{tabular}

\subsection{p2 Aufgabe S. 60}
Der Chef des Ski- und Snowboardverleihs Winterhart speichert den Verleih seiner Ausr"ustung in einem Data Warehouse.
Mithilfe von OLAP m"ochte er einen besseren "Uberblick dar"uber bekommen, welche Ausr"ustung an welchen Standorten in welchen Zeitr"aumen verliehen wird.\\

\noindent
$a)$ In einem Buch hat der Chef des Ski- und Snowboardverleihs etwas "uber die OLAP-Operationen Slice, Dice und Roll-Up gelesen.
Erl"autern Sie bitte allgemein das Ziel dieser OLAP-Operatoren.

\begin{quote}
    \textbf{Slice/Dice}\\
    Ebenen (Slices) bzw. Unterw"urfel (Dices) werden in das Blickfeld des Betrachters geholt.\\
    \textbf{Drill-Down/Roll-Up}\\
    Stufenweise detaillierte bzw. aggregiertere Betrachtungsweise.\\
\end{quote}

\noindent
$b)$ Der Chef des Ski- und Snowboardverleihs zeigt Ihnen folgenden Ausschnitt aus seiner Datawarehouse-Tabelle "`Verleih"':\\

\rowcolors{0}{LightGray}{White}
\begin{tabular}{ l l l l }
    \rowcolor{LightSlateGrey}
    \textbf{Ausr"ustung} & \textbf{Standort} & \textbf{Monat} & \textbf{Anzahl}\\
    Ski         & Oberstdorf                & 12/2010   & 840\\
    Ski         & Garmisch-Partenkirchen    & 12/2010   & 960\\
    Ski         & Oberstdorf                & 01/2011   & 530\\
    Ski         & Garmisch-Partenkirchen    & 01/2011   & 420\\
    Snowboard   & Oberstdorf                & 12/2010   & 900\\
    Snowboard   & Garmisch-Partenkirchen    & 12/2010   & 950\\
    Snowboard   & Oberstdorf                & 01/2011   & 600\\
    Snowboard   & Garmisch-Partenkirchen    & 01/2011   & 550\\
\end{tabular}

\bigskip

\noindent
Bitte zeigen Sie auf welche Ergebnisse der Chef des Ski- und Snowboardverleihs erh"alt, wenn er folgendes SQL-Statement formuliert.\\

\lstset{style=customSQL}
\begin{lstlisting}
SELECT                  Ausruestung
                        , Monat
                        , Sum(Anzahl) AS Gesamtanzahl
FROM                    Verleih
GROUP BY GROUPING SETS  ((Ausruestung, Monat)
                            , (Ausruestung)
                            , ())
;
\end{lstlisting}

\bigskip

\rowcolors{0}{LightGrey}{White}
\begin{tabular}{ l l l }
    \rowcolor{LightSlateGray}
    \textbf{Ausr"ustung} & \textbf{Monat} & \textbf{Gesamtanzahl}\\
    Ski         & 12/2010   & 1800\\
    Ski         & 01/2010   & 950\\
    Snowboard   & 12/2011   & 1850\\
    Snowboard   & 01/2011   & 1150\\
    Ski         &           & 2750\\
    Snowboard   &           & 3000\\
\end{tabular}

