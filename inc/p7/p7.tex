
\chapter{PIM Praxisteil 07}

\section{Modellierung Gesch"aftsprozesse}

\subsection{p7 Aufgabe S. 4}

In einer Produktionsst"atte tritt ein Fehler bei einer Maschine auf.
Daraufhin erstellen Mitarbeiter aus der Produktion eine Fehlermeldung.\\
Anhand dieser Fehlermeldung suchen Mitarbeiter aus der Instandsetzungsabteilung die Ursache f"ur den Fehler.
Nachdem die Ursache gefunden wurde, versucht die Instandzetzungsabteilung den Fehler zu beheben.
Informationen zur Fehlerbehubung werden in einem Dokument ("`Reperaturversuch"') erfasst.\\
Daraufhin validieren Mitarbeiter aus der Produktion die Fehlerbehebung.
Wenn der Fehler erfolgreich behoben ist, wird der Prozess beendet.
Ist der Fehler nicht bollst"andig behoben, beginnt die Instandzetzungsabteilung erneut mit der Fehlersuche.\\
Nach f"unf nicht erfolgreichen Versuchen wird eine neue Maschine beschafft und der Prozess beendet.\\

\includegraphics[scale=0.35]{./inc/p7/p7_maschinenfehler}


\subsection{p7 S. 9}

Ein Callcenter, das f"ur ein Kino arbeitet, nimmt Reservierungsw"unsche von Kunden entgegen.
Die Kunden geben dabei ihren Wunschtermin sowie den Wunschfilm vor.
Daraufhin sucht der Callcenter-Agent eine geeignete Vorstellung heraus.\\
Wenn eine Veranstaltung gefunden wurde, erfragt der Callcenter-Agent vom Kunden die Zahl der ben"otigten Katen.\\
Falls in der Vorstellung gen"ugend Pl"atze frei sind, wird die Reservierung durchef"uhrt und eine Reservierungsbest"atigung per eMail versendet.\\
Der Einfachheit halber wird in diesem Szenario der Prozess abgebrochen, wenn keine geeignete Veranstaltung gefunden wurde oder nicht gen"ugend Pl"atze zur Verf"ugung stehen.\\


