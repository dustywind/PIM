

\chapter{PIM Praxisteil 04}

\section{XQuery}

\subsection{p4 Aufgabe S. 3}

Der Online Sportschuhversand Punidas24 verwaltet seinen Produktkatalog mithilfe von XML.
Nachfolgend ist ein Ausschnitt aus dem Produktkatalog dargestellt.\\

\lstset{style=customXML}
\begin{lstlisting}
<?xml version='1.0' encoding='ISO-8859-1'?>
<!DOCTYPE Sportschuhkatalog SYSTEM 'Sportschuhkatalog.dtd'>
<Sportschugkatalog>
    <Sportschuh>
        <ID>345633</ID>
        <Modell Kategorie='Damen'>Speedy33</Modell>
        <Hersteller>Puma</Puma>
        <Beschreibung>Speedy33 ist aus Glattleder gefertigt.</Beschreibung>
        <Basisfarbe>gruen</Basisfarbe>
        <Preis>88,90</Preise>
    </Sportschuh>
    <Sportschuh>
        <ID>419984</ID>
        <Modell Kategorie='Herren'>Absolut55</Modell>
        <Hersteller>Adidas</Hersteller>
        <Basisfarbe>rot</Basisfarbe>
        <Preise>49,90</Preise>
    </Sportschuh>
</Sportschuhkatalog>
\end{lstlisting}

\noindent
Die Marktforschungsabteilung von Pumidas24 prognostiziert eine steigende Nachfrage nach gr"unen Sportschuhen f"ur die n"achsten Monate.
Pumidas24 m"ochte daher seinen Bestand an gr"unen Sportschuhen erh"ohen.
Bitte formlulieren Sie mithilfe von XQuery eine Abfrage, die aus dem Produktkatalog alle Identifikationsnummern (ID) von gr"unen Sportschuhen selektiert.
Bitte beachten Sie, dass der Produktkatalog in einer Datei namens "`Sportkatalog.xml"' gespeichert ist.\\

\lstset{language=PHP}
\begin{lstlisting}
for $schuh in doc("Sportschuhkatalog.xml")/Sportschuhkatalog/Sportschuh

where $schuh/Basisfarbe = "rot"

return $schuh/ID
\end{lstlisting} %$ // correct the syntaxhighlithing within vim

\section{Content Management Systeme (CMS)}

\subsection{p4. Aufgabe S. 5}
Das Online-Sport-Portal Live.de m"ochte einen kostenpflichtigen Premium-Bereich schaffen, in dem Seiten mit Sportnachrichten zur Verf"ugung gestellt werden.
Abh"angig vom Interessenprofil der Mitglieder sollen Berichte zu Ereignissen verschiedener Sportarten indiviudalisiert werden.\\

\noindent
$a)$ Bitte erl"autern Sie die Unterschiede zwischen dem \textbf{Layoutorientierten} und dem \textbf{Contentorientierten} Verfahren zur Erstellung von Medienprodukten.
Begr"unden Sie bitte, welches Verfahren sie Live.de empfehlen w"urden.\\

\begin{quote}
    Beim \textbf{layoutorientierten} Verfahren einigt man sich auf ein bestimmtes Layout, welches anschliessend nicht mehr ver"andert wird.
    Stattdessen wird jeder neue Beitrag (Content) an dieses Layout angepasst.\\
    Das \textbf{contentorientierte} Verfahren hingegen passt das Layout dem Content an.\\

    \noindent
    F"ur Webseiten wie Live.de ist das contentorientierte Verfahren besser geeignet.
\end{quote}




\noindent
$b)$ Bitte zeigen Sie die Unterschiede zwischen der \textbf{statischen} und \textbf{dynamischen} erzeugung von Webseiten auf.
Welches Verfahren ist f"ur Live.de besser geeignet?
Bitte begr"unden Sie ihre Antwort.\\
\begin{quote}
    W"ahrend bei statischen Webseiten der Inhalt immer gleich ist, bis ihn der Webmaster ver"andert, k"onnen dynamische Webseiten  mit dem Benutzer interagieren.\\
    Das bedeutet, dass z.B. auf Suchanfragen des Nutzers oder auf seine Eingaben vom Webserver reagiert wird.\\

    \noindent
    Da auf Live.de individualisierte Inhalte dargestellt werden sollen ist unbedingt einem dynamisches Verfahren anzuwenden.
\end{quote}



\section{Dokumentenretrieval}

\subsection{p4 Aufgabe S. 8}

In der Jobb"orse job24 befinden sich 40 Praktikumspl"atze im Bereich Wirtschaftsinformatik und 60 Praktikumspl"atze im Bereich Informatik.
Die Suchanfrage "`Praktikum Wirtschaftsinformatik"' liefert 30 Praktikumspl"atze, von denen jedoch 10 nicht relevant sind.
Bitte berechnen Sie die \textbf{Precision} und \textbf{Recal} der Suchmaschine.\\

\begin{quote}
    $Precision = N_{relgef} \div N_{gef} = 20 \div 30 = \frac{2}{3}$\\
    $Recall = N_{relgef} \div N_{relgesamt} = 20 \div 60 = \frac{1}{3}$\\

    {\tiny $N_{rel}$ = Anzahl der relevanten Ergebnisse}\\
    {\tiny $N_{gef}$ = Anzahl der gefundenen Ergebnisse}
\end{quote}


\subsection{p4 Aufgabe S. 10}
Die Gesch"aftskunden der Arbeitsvermittlungsagentur job4u bem"angeln, dass die webbasierte Stichwortsuche nach geeigneten Bewerbern l"angst nicht alle relevante Kandedaten zutate f"ordert.\\

\noindent
$a)$ Bitte erl"autern Sie den \textbf{Unterschied zwischen Recall und Precision}.
Gehen sie hierbei auch auf den Zusammenhang zwischen diesen beiden Gr"o"sen ein.\\

\begin{quote}
    \textbf{Precision} ist das Ma"s f"ur den Anteil der relevanten gefundenen Dokumente an den insgesamt gefundenen Dokumenten.\\
    $Precision = N_{relgef} \div N_{gef}$\\

    \textbf{Recall} ist das Ma"s f"ur die relevanten Dokumente, die gefunden werden, im Verh"altnis zu allen (gefundenen und nicht gefundenen) relevanten Dokumenten.\\
    $Recall = N_{relgef} \div N_{rel}$\\


    \textbf{\#\#\# TODO \#\#\#\\Zusammenhang zwischen Precision + Recall}\\
    \textbf{Vermutung:}\\
    Die Verbesserung des Recalls geht immer zu lasten der Precision und vice versa.

\end{quote}

\noindent
$b)$ Welche dieser beiden Gr"o"sen muss beim Online-Portal von Job4u verbessert werden?\\
\begin{quote}
    Es muss die Gr"o"se "`Recall"' verbessert werden.
\end{quote}

\noindent
$c)$ Bitte nennen und erl"autern Sie einen Retrieval-Ansatz Ihrer Wahl, der dem beim job4u identifizierten Problem entgegenwirklt.\\
\begin{quote}
    Es sollte ein Text-Retrieval-Ansatz verfolgt werden.\\
    Dabei ist beispielsweise die Fuzzy-Suche zu empfehlen.
    Fuzzy-Suche bedeutet, dass das Suchkriterium nicht zu $100\%$ mit dem Ergebnis "ubereinstimmen muss, sondern leicht davon abweichen darf.\\
    In der Praxis hat dies den Vorteil, dass die Suche nach "`Informatik"' auch Ergebnisse aus der Menge der "`Wirtschaftsinformatiker"' und "`BWL mit Schwerpunkt Informatik"' liefert. Dadurch werden mehr relevante Eintr"age gefunden.
\end{quote}


\subsection{p4 Aufgabe S. 12}
Sie sind mit dem Aufbau einer Wissensplattform f"ur eine Anwaltskanzlei betraut.
An die neu entwickelte Suchmaschine wird die Suchanfrage "`\textbf{erg"anzung zum aktiengesetz im jahre 2002}"' gesendet.
Es wurden bereits "uber  10.000 Dokumente in das Repository der Suchmaschine aufgenommen.
Unter anderem befinden sich darin die Dokumente $A$, $B$, und $C$ mit den folgenden Worth"aufigkeiten:\\

\noindent
$A$: bundestag ($2\times$), 2002 ($3\times$), aufsichtsrat ($4\times$), aktiengesetz ($1\times$), gr"undung ($2\times$)\\
$B$: gesetzgeber ($1\times$), gesch"aftspolitik ($1\times$), erg"anzung ($2\times$), aktiengesetz ($4\times$)\\
$C$: 2002 ($1\times$), rechnungslegung ($3\times$), aktiengesetz ($1\times$), erg"anzung ($1\times$), konzern ($2\times$)\\

\noindent
Um Suchergebnisse nach ihrer Relevanz zu sortieren, werden Ranking-Algorithmen wie "`Simple Match"' oder "`Weighted Match"' eingesetzt.\\

\noindent
In welcher Reihenfolge pr"asentiert die Suchmaschine die dokumente $A$, $B$, und $C$, wenn der Simple-Match-Algorithmus eingesetzt wird?
Bitte geben Sie die resultierende Gewichtung der Dokumente mit an.\\

\rowcolors{1}{LightGray}{White}
\begin{tabular}{ l l l l l l }
    \rowcolor{LightSlateGrey}
    \textbf{lex} & \textbf{erg"anzung} & \textbf{aktiengesetz} & \textbf{jahr} & \textbf{2002} & \textbf{Gewicht}\\
    ret     & 1 & 1 & 1 & 1 & 4\\
    docA    & 0 & 1 & 0 & 1 & 2\\
    docB    & 1 & 1 & 0 & 0 & 2\\
    docC    & 1 & 1 & 0 & 1 & 3\\
\end{tabular}

\bigskip
\noindent
Reihenfolge:\\
$\Rightarrow$ docC, docB, docA\\

\subsection{p4 Aufgabe S. 15}
Das Polizeipr"asidium Mittelfranken verf"ugt neuerdings "uber ein hochmodernes Bildretrievalsystem zur Recherche in der Verbrecherkartei.
Die indexierten Digitalaufnahmen und eingescannten Fotografien stehen den Beamten f"ur Muster- und Metasuchanfragen zur Verf"ugung.\\

\noindent
Erl"autern Sie bitte den Unterschied zwischen Muster- und Metasuche.
Nennen Sie je zwei Beispiele f"ur entsprechende Suchmerkmale bei einer Bildrecherche zum Retrieval von Gesichtern.

\begin{quote}
\end{quote}



































